\documentclass[a4paper,twoside,12pt]{article}
%\usepackage [reqno] {amsmath}
\usepackage{amsfonts,amstext}
\usepackage{amsmath}
\usepackage{german}
\usepackage{fullpage}

\newcommand{\ZETTELNUMMER}{1}
\newcommand{\ABGABEDATUM}{am 27. Oktober~2017 bis 10 Uhr in die 
  jeweiligen Tutorenf\"acher}

\newcounter{AUFGNR}
\setcounter{AUFGNR}{1}
\newcommand{\AUFGABE}[2]{\vspace{0.3cm}\item[Aufgabe \arabic{AUFGNR}]\stepcounter{AUFGNR} #1\hfill\emph{#2}}


\newcommand{\floor}[1]{\left\lfloor{#1}\right\rfloor}
\newcommand{\ceil}[1]{\left\lceil{#1}\right\rceil}
\newcommand{\half}[1]{\frac{#1}{2}}



\renewcommand{\labelenumi}{(\alph{enumi})}


\begin{document}
\pagestyle{empty}
\hrule\medskip
\rule{0ex}{0ex}\\[-1ex]
\ZETTELNUMMER. Aufgabenblatt zur Vorlesung

\smallskip
\noindent
\large
\textbf{Algorithmen, Datenstrukturen und Datenabstratkion}\hfill WS 17/18 \\[0.5ex]
\normalsize
Wolfgang Mulzer, Katharina Klost

\medskip\hrule

\smallskip
\noindent
\textbf{Abgabe} \ABGABEDATUM

\vskip 0.5cm

\begin{description}

\AUFGABE{Rekursionsgleichungen}{10 Punkte}

L\"osen Sie die folgenden Rekursionsgleichungen.
Beweisen Sie jeweils die Richtigkeit Ihrer L\"osung
(z.B. durch vollst\"andige Induktion).
\begin{enumerate}
  \item
    \[
       T(1) = 0; \quad T(n) = 
       T\Bigl(\left\lfloor \frac{n}{2}\right\rfloor\Bigr) +
                 T\Bigl(\left\lceil \frac{n}{2} \right\rceil\Bigr) + n, 
		 \text{ f\"ur } n > 1.
    \]
    Sie d\"urfen annehmen, dass $n$ eine Zweierpotenz ist.
  \item 
    \[
      S(1) = 1; \quad S(n) = \sum_{i=1}^{n-1} i \cdot S(i), 
        \text{ f\"ur } n > 1.
    \]
\end{enumerate}


\AUFGABE{Die Ungleichung vom arithmetischen und geometrischen Mittel}{10 Punkte}


Seien $x_1, x_2, \ldots, x_n \geq 0$. Beweisen Sie
die Ungleichung
\[
P(n): x_1 x_2 \ldots x_n
\leq\left(\frac{x_1+\dots+x_n}{n}\right)^n,
\]
indem Sie folgende Teilschritte bearbeiten:
\begin{enumerate}
  \item $P(2)$ ist eine wahre Aussage.

     \emph{Hinweis}: $(x_1 - x_2)^2 \geq 0$.
  \item
     Wenn $P(n)$ gilt, dann gilt auch $P(n-1)$.

     \emph{Hinweis}: Wie muss man $x_n$ w\"ahlen, damit 
     sich der Wert der rechten Klammer nicht \"andert, wenn
     $x_1, \ldots, x_{n-1}$ schon feststehen?

   \item Aus $P(2)$ und $P(n)$ folgt $P(2n)$.
   \item Folgern Sie nun die Ungleichung. 
\end{enumerate}

\AUFGABE{Manipulation elementarer Funktionen}{10 Punkte}

Finden Sie Paare von \"aquivalenten Termen und formen Sie diese
schrittweise ineinander um. Geben Sie die verwendeten Regeln an.
\[
 \log_a\Bigl(n^{\log_b a}\Bigr), \sqrt[b]{\frac{a^n}{a^m}},
 b^{n \log a}, \log_b n, a^{\frac{n-m}{b}}, n(\log a + \log b),
 \log(a^nb^n),a^{(\log b^n)}.
\]
\iffalse
\AUFGABE{Die inverse schnelle Fouriertransformation (freiwillig)}
{10 Zusatzpunkte}

\begin{enumerate}
\item Zeigen Sie mit einer Beweismethode Ihrer Wahl,
dass f\"ur alle $x \not= 1$ und alle $n \geq 0$ gilt:
\[
\sum_{i=0}^{n} x^i = \frac{x^{n+1}-1}{x-1}.
\]
\item Im Folgenden sei $n+1$ eine Zweierpotenz und
$\omega \in \mathbb{C}$ eine 
\emph{komplexe} primitive $(n+1$)-te Einheitswurzel
(d.h., $\omega^{n+1} = 1$ und $\omega^k \neq 1$, f\"ur
$k=1, \dots, n$).
Sei $p(x) = a_0 + a_1 x + \dots + a_{n} x^{n}$
ein Polynom vom Grad $n$.

In der Vorlesung haben Sie gesehen,
wie man schnell die Werte $p(\omega^0)$,
$p(\omega^1)$, $\ldots$, $p(\omega^{n})$ ausrechnen kann.
Sei $A = (a_{ij})$ die $(n+1) \times (n+1)$ Matrix
mit $a_{ij} = \omega^{(i-1)(j-1)}$.

Zeigen Sie, wie man das Ausrechnen von
$p(\omega^0)$,
$p(\omega^1)$, $\ldots$, $p(\omega^{n})$
als Multiplikation von
$A$ mit einem Vektor darstellen kann. Das hei\ss{}t, finden
Sie einen Spaltenvektor $v \in \mathbb{C}^{n+1}$ mit
\[
\left(\begin{matrix} p(\omega^0) \\ p(\omega^1) \\ \vdots \\ p(\omega^{n}) 
\end{matrix} \right)= A\cdot v.
\]
\item
  Sei $B = (b_{ij})$ die $(n+1) \times (n+1)$ Matrix mit
  $b_{ij} = \frac{1}{n+1}\omega^{-(i-1)(j-1)}$.
  Zeigen Sie, dass $B \cdot A = I$ ist, wobei $I$ 
  auf der Diagonale nur Einsen und in den
  sonstigen Eintr\"agen nur Nullen enth\"alt (die \emph{Einheitsmatrix}). 
  Insbesondere ist
  $B \cdot A \cdot w = w$, f\"ur alle $w \in \mathbb{C}^{n+1}$.

  \emph{Hinweis:} Verwenden Sie (a) und beachten Sie, dass $\omega$
      eine $(n+1)$-te Einheitswurzel ist.

\item Sei nun $p(x)$ ein Polynom vom Grad $n$,
  von dem nur die Werte
  $p(\omega^0)$, $p(\omega^1)$, $\ldots$, $p(\omega^{n})$
  bekannt sind.
  Beschreiben Sie ein Verfahren, um in
  $O(n \log n)$ Operationen die Koeffizienten
  von $p(x)$ zu erhalten. 

  \emph{Hinweis:} Aufgrund von (c) m\"ussen wir die Matrix $B$
     mit einem geeigneten Spaltenvektor multiplizieren. Zeigen Sie,
     dass dies der Auswertung eines geeigneten Polynoms an den Stellen
     $\omega^{-0}$, $\omega^{-1}$, $\ldots$, $\omega^{-n}$ entspricht.
     Gehen Sie analog zur Vorlesung vor, um dies in
     $O(n \log n)$ Operationen zu bewerkstelligen.

 \item 
    Seien $p(x)$ und $q(x)$ zwei Polynome vom Grad $n$
    in Koeffizientenform. 
    Skizzieren Sie ein Verfahren, wie man in $O(n\log n)$ Operationen
    das Produkt $r(x) = p(x) \cdot q(x)$
    in Koeffizientenform berechnen kann 
    (\emph{Achtung}: Das Produkt hat Grad $2n$.).
\end{enumerate}
\fi
\end{description}
\end{document}
